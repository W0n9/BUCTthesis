\documentclass[zihao=-4]{ctexart}
\usepackage{bucttaskbook}

\buctsetup{
	%%%%% 各选项之间不要留有空行,并以西文逗号“,”分隔 %%%%%
	% 论文的中文标题
	ctitle		= {BUCTthesis 任务书示例文档},
	% 作者姓名
	cauthor		= {张三},
	% 班级
	class		= {某某1024},
	% 学院
	school		= {材料科学与工程学院},
	% 专业名称
	major		= {高分子材料与工程},
	% 导师的姓名与职称
	supervisor	= {李四教授},
	% 专业负责人姓名
	msupervisor	= {王五},
}

\begin{document}
\begin{taskbook}
    \taskinfo		% or use `\taskinfo*' for less lines.
    
    \taskitem		% 1.设计(论文)的主要任务及目标
    大学之道,在明明德,在亲民,在止于至善。知止而后有定;
    定而后能静;静而后能安;安而后能虑;虑而后能得。
    物有本末,事有终始。知所先后,则近道矣。
    
    古之欲明明德于天下者,先治其国;欲治其国者,先齐其家;
    欲齐其家者,先修其身;欲修其身,先正其心;欲正其心者,先诚其意;
    欲诚其意者,先致其知;致知在格物。物格而后知至;知至而后意诚;
    意诚而后心正;心正而后身修;身修而后家齐;家齐而后国治;
    国治而后天下平。
    
    自天子以至于庶人,壹是皆以修身为本。其本乱而未治者否矣。
    其所厚者薄,而其所薄者厚,未之有也。此谓知本,此谓知之至也。
        
    \taskitem		% 2.设计(论文)的基本要求和内容
    古之学者必有师。师者,所以传道受业解惑也。人非生而知之者,孰能无惑?
    惑而不从师,其为惑也,终不解矣。生乎吾前,其闻道也固先乎吾,吾从而
    师之;生乎吾後,其闻道也亦先乎吾,吾从而师之。吾师道也,夫庸知其年
    之先後生於吾乎!是故无贵无贱无长无少,道之所存,师之所存也。
        
    \taskitem		% 3.主要参考文献
        \begin{bibenumerate}
            \item 北京化工大学教务处. 本科生毕业设计(论文)撰写规范 [EB/OL]. 2018[2020-04-08]. \url{https://jiaowuchu.buct.edu.cn/2018/1009/c515a22046/page.htm}.
            \item 刘海洋. \LaTeX\ 入门 [M]. 北京 : 电子工业出版社, 2013.
            \item MITTELBACH F, GOOSSENS M, BRAAMS J, et al. The \LaTeX\ Companion[M]. 2nd ed. Reading, Massachusetts : Addison-Wesley, 2004.
            \item 
        \end{bibenumerate}
        
    \taskitem		% 4.进度安排
        \begin{table}[H]
            \centering
            \begin{tabularx}{.95\textwidth}{p{1.5em}|X|p{6em}}
                \hline
                        & 	设计(论文)各阶段名称		  &		起止日期	\\\hline
                    1	& 								&				\\\hline
                    2	&								&				\\\hline
                    3	&								&				\\
                \hline
            \end{tabularx}
        \end{table}
    \end{taskbook}
\end{document}